\documentclass[12pt, onecolumn, a4paper]{article}
\usepackage[T1]{fontenc}
\usepackage[utf8]{inputenc}
\usepackage{datetime} %To get the month name in words
\usepackage{fontspec}
\setmainfont{Carlito}
\usepackage{amsfonts,amsmath}
\usepackage[lmargin=1in, rmargin=1in, tmargin=1in, bmargin=1in]{geometry}
\usepackage{array}
\usepackage{hyperref}
\hypersetup{
	colorlinks = true, %Colours links instead of ugly boxes
	urlcolor = black, %Colour for external hyperlinks
	linkcolor = blue, %Colour of internal links
	citecolor = red %Colour of citations
}
\usepackage{authblk}
\usepackage{fancyhdr}
\pagestyle{fancy}
\fancyhf{}
\rhead{\thepage}
\makeindex
\date{\empty}
\author[1]{Vibishan B.}
\author[1,*]{Milind G. Watve}
\affil[1]{\small Department of Biology, Indian Institute of Science Education and Research, Pune}
\affil[*]{\small Corresponding author: milind@iiserpune.ac.in}

\title{Context-dependent selection as the keystone in the somatic evolution of cancer}

\begin{document}
\maketitle

Somatic evolution of cancer involves a series of mutations, and attendant changes, in one or more clones of cells. Unlike a ``bad luck'' type model, the notion of clonal expansion adds competition-driven selection to the supposedly random process of somatic mutagenesis, with the implicit assumption that any mutation leading to partial loss of regulation of cell proliferation will give a selective advantage to the mutant. However, a number of experiments show that an intermediate pre-cancer mutant has only a conditional selective advantage; given that tissue microenvironmental conditions differ across individual organisms, this selective advantage to a mutant should be widely distributed over the population of organisms. We evaluate three models, namely ``bad luck'', context-independent, and -dependent selection, in a comparative framework, on their ability to predict patterns in total incidence, age-specific incidence, and their ability to explain Peto’s paradox. Results show that context dependence is necessary and sufficient to explain observed epidemiological patterns, and that cancer incidence is largely selection-limited, as opposed to the mutation-centric, ``bad luck'' view. A wide range of physiological, genetic and behavioural factors influence the tissue micro-environment, and could therefore be the source of this context dependence in somatic evolution of cancer. The identification and targeting of these micro-environmental factors that influence the dynamics of selection offer new possibilities for cancer prevention. Our work also seeks to renew interest in the comparative evaluation framework, whose application has seen a lull in cancer literature, despite the possibilities of rejection it offers for potential theories of carcinogenesis.

\end{document}